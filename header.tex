\documentclass[accentcolor=tud1b,longdoc,nopartpage,table,oneside,bibtotoc, liststotoc,openright,colorbacktitle,inverttitle,numbersubsubsec,noresetcounter,11pt,noheadingspace,numbers=noenddot,article,parskip=half]{tudreport}

% Babel-Paket f. neue deutsche Rechtschreibung
\usepackage[ngerman, english]{babel}
\usepackage{graphicx}
% Eingabekodierung auf latin1 setzen
\usepackage[utf8]{inputenc}


% Font-Encoding auf T1 setzen
\usepackage[T1]{fontenc}
% footmisc behebt u.a. Probleme mit Fu�noten in Abschnittstiteln
\usepackage[stable]{footmisc}

% Einbinden von Grafiken erm�glichen
\usepackage{graphicx}

\usepackage{newunicodechar}
\newunicodechar{ß}{\ss}
% Paket xtab erm�glicht Umbrechen von langen Tabellen
% \usepackage{xtab}

% picins erlaubt das Umflie�en von Abbildungen durch Text
% Untenstehendes renewcommand behebt den picins-bug, dass Abbildungen
% nicht im Abbildungsverzeichnis auftauchen
\usepackage{picins}
\makeatletter
\renewcommand\piccaption{\@dblarg{\@piccaption}}
\makeatother

\usepackage{verbatim}

% Einr�ckung bei Zeilenumbruch vermeiden
\setlength{\parindent}{0pt}
\setlength{\parskip}{9pt}
% Tiefe des Inhaltsverzeichnis (paragraphen werden mit aufgenommen)
\setcounter{secnumdepth}{4}
\setcounter{tocdepth}{4}
%Überschriften 
\addtokomafont{subsection}{\normalsize\bf\sffamily}
\addtokomafont{subsubsection}{\normalsize\bf\sffamily}
\addtokomafont{paragraph}{\normalsize\bf\sffamily}


%Silbentrennung von zusammengesetzten w�rtern
%\usepackage[ngerman=ngerman-x-latest]{hyphsubst}
% Paket setspace erlaubt Umschalten auf 1.5fachen Zeilenabstand
\usepackage{setspace}


%%%%%%%%%%%%%%%%%%%%%%%%%%%%%%%%%%%%%%%%%%%%%%%%%%%%%%%%
%% Anpassungen von Literaturverzeichnis und Zitierweise%
\usepackage[hyphens]{url}
\usepackage[commabeforerest,authorformat=year,see,dotafter=bibentry,pages=format]{jurabib} % , ibidem=strict 
%\AddTo\bibsgerman{%
%\renewcommand*{\ibidemname}{ebd.}
%\renewcommand*{\ibidemmidname}{ebd.}
%}
%Trennzeichen zwischen Autoren im Zitat

%\AddTo\bibsgerman{%
%\def\jbpagename{S.}%
%\def\jbpagesname{S}%
%}

\renewcommand*{\jbbtasep}{ \& }
\renewcommand*{\jbbfsasep}{; }
\renewcommand*{\jbbstasep}{; }

%Trennzeichen zwischen Autoren im Literaturverzeichnis
\renewcommand*{\bibbtasep}{ \& }
\renewcommand*{\bibbfsasep}{; }
\renewcommand*{\bibbstasep}{ \& }

%Trennzeichen zwischen Herausgebern im Literaturverzeichnis
\renewcommand*{\bibbtesep}{; }
\renewcommand*{\bibbfsesep}{; }
\renewcommand*{\bibbstesep}{; }

%Unterdr�ckt, dass bei mehr als drei Autoren im Literaturverzeichnis
%mit "et al." abgek�rzt wird --> myjureco.bst-Datei wird zus�zlich ben�tigt!
\makeatletter
\def\jb@use@fullcite{%
\jbauthorfont{\jb@@author}\normalfont{\jbhowsepbeforetitle}\jb@@fulltitle}%
\makeatother

%In: erscheint vor dem Titel von Zeitschriften, Konferenzbeitr�gen, Sammelwerken
%und Herausgeberb�nden
%\AddTo\bibsall{\def\incollinname{\textbf{In:}}}
\renewcommand{\bibbtsep}{In: }
\renewcommand*{\bibjtsep}{In: }

%Vor- und Nachname des Herausgebers werden nicht fett gedruckt
\renewcommand*{\biblnfont}{} 
\renewcommand*{\bibfnfont}{}

%�ndert bei urldate das Pr�fix von "Zugriff am" zu "Abruf am"
\AddTo\bibsgerman{\renewcommand*{\urldatecomment}{Last accessed: }}

%Setzt ein Komma zwischen der URL und "Abruf am"
\renewcommand*{\bibbudcsep}{, }

%Entfernt die Zeichen vor und nach der URL-Angabe im Literaturverzeichnis
\renewcommand*{\biburlprefix}{}
\renewcommand*{\biburlsuffix}{}

%Entfernt das Komma zwischen Jahrgang und Ausgabe
%und setzt die Ausgabe in Klammern
\renewcommand*\artnumberformat[1]{\unskip\space (#1)}

%Entfernt das Komma zwischen Adresse und Verlag
%und setzt daf�r ein Leerzeichen. Dies ist n�tig,
%da die Reihenfolge von Adresse und Verlag in myjureco vertauscht wird
%und kein Zeichen nach der Adresse erscheinen soll.
\renewcommand*\bpubaddr{}
\renewcommand*{\bibtfont}{\textit}
% Check if this gives errors
\renewcommand*{\bibapifont}{\textit}
\renewcommand*{\bibatsep}{,}
\renewcommand*{\ajtsep}{,}

%%%%%%%%%%%%%%%%%%%%%%%%%%%%%%%%%%%%%%%%%%%%%%%%%%%%%%%%


\usepackage{hyperref}
%Erm�glicht Hyperlinks zwischen Textstellen und zu externen Dokumenten
%% breaklinks=true/false: Gibt an, ob Links umgebrochen werden d�rfen.
%% linktocpage=true/false: im Inhaltsverzeichnis sind nur die Seitenzahlen
%% links, nicht der Text
%% colorlinks=true/false: Links werden eingef�rbt (Farben werden mit
%% linkcolor, anchorcolor \dots festgelegt)
%% linkcolor=Farbe: Farbe des verlinkten Textes, Dokument-interne Links
%% citecolor=Farbe: Farbe des verlinkten Textes, Links zum
%% Literaturverzeichnis
%% filecolor=Farbe: Farbe des verlinkten Textes, Links auf lokale Dateien
%% urlcolor=Farbe: Farbe des verlinkten Textes, externe URLs
%% frenchlinks=true/false: Links werden als smallcaps, anstatt farbig
%% dargestellt.
%% breaklinks=true/false: Gibt an, ob Links umgebrochen werden d�rfen.
\hypersetup{%
bookmarks=true,
unicode=false,
pdftoolbar=true,
pdfmenubar=true,
pdffitwindow=true,
pdfstartview={FitH}
  filecolor=black,
  breaklinks=true,
  colorlinks=true,
  citecolor=black,
  urlcolor=black,
  linkcolor=black,
  pdfpagemode=UseThumbs,
  pdftitle=Mustertitel,
  pdfauthor=Max Mustermann,
  pdfsubject=Musterthema,
  %pdfkeywords=xy
}




% Eigene Pakete hier 
\usepackage{rotating}
\usepackage{booktabs,dcolumn}
\usepackage{longtable}
\usepackage{paralist}
\usepackage{dcolumn}
\usepackage{nameref}
\usepackage[font=footnotesize ,font=bf]{caption}
\usepackage{subcaption}
%\usepackage{tex4ht}
\usepackage{listings}
\usepackage{etex}
\usepackage{tikz}
	\usetikzlibrary{patterns}
\usepackage{pgfplots}
%	\pgfplotsset{compat=1.11}
%	\usepgfplotslibrary{fillbetween}
\usepackage{icomma}
\usepackage{float}
\usepackage{amsmath}
\usepackage{todonotes}
\usepackage{footnote}
\usepackage{multirow}
% \usepackage{helvet}
% \renewcommand{\familydefault}{\sfdefault}
% \fontfamily{helvet}\selectfont
% \renewcommand{\familydefault}{\rmdefault}
% \setmainfont{helvet}
% \setsansfont{helvet}
% \setmonofont{helvet}
% \setmathfont{helvet}
\usepackage{fontspec}
\setmainfont[Ligatures=TeX]{charter}
\setsansfont[Ligatures=TeX]{FrontPage-Medium.ttf}
% \setmonofont{FrontPage-Regular.ttf}
\selectfont
\usepackage[toc,nopostdot,nonumberlist]{glossaries}
\makeglossaries
\loadglsentries[main]{glossaries}

% add citations from image captions
\makesavenoteenv{figure}

% todonotes setup
\newcommand{\addcitation}[1]{\todo[color=yellow,inline]{Missing citation: #1}}
\newcommand{\outline}[1]{\todo[color=green,inline]{Outline: #1}}
\newcommand{\structure}[1]{\todo[color=pink,inline]{Structure: #1}}

% display table caption on top
\floatstyle{plaintop}
\restylefloat{table}

% align caption to the left
\captionsetup{justification=raggedright,singlelinecheck=false}

\colorlet{punct}{red!60!black}
\definecolor{background}{HTML}{EEEEEE}
\definecolor{delim}{RGB}{20,105,176}
\colorlet{numb}{magenta!60!black}

\lstdefinelanguage{json}{
    basicstyle=\normalfont\ttfamily,
    numbers=left,
    numberstyle=\scriptsize,
    stepnumber=1,
    numbersep=8pt,
    showstringspaces=false,
    breaklines=true,
    frame=lines,
    backgroundcolor=\color{background},
    literate=
     *{0}{{{\color{numb}0}}}{1}
      {1}{{{\color{numb}1}}}{1}
      {2}{{{\color{numb}2}}}{1}
      {3}{{{\color{numb}3}}}{1}
      {4}{{{\color{numb}4}}}{1}
      {5}{{{\color{numb}5}}}{1}
      {6}{{{\color{numb}6}}}{1}
      {7}{{{\color{numb}7}}}{1}
      {8}{{{\color{numb}8}}}{1}
      {9}{{{\color{numb}9}}}{1}
      {:}{{{\color{punct}{:}}}}{1}
      {,}{{{\color{punct}{,}}}}{1}
      {\{}{{{\color{delim}{\{}}}}{1}
      {\}}{{{\color{delim}{\}}}}}{1}
      {[}{{{\color{delim}{[}}}}{1}
      {]}{{{\color{delim}{]}}}}{1},
}

% Anpassungen der R�nder an die Vorgaben des Lehrstuhls
% BACKUP \geometry{left=3cm, right=2cm, top=1.5cm, bottom=2cm}
\geometry{left=3cm, right=1.9cm, top=1.7cm, bottom=1.9cm,headsep=5mm, nomarginpar, includeall}

\hyphenation{In-for-ma-ti-on Apps}
%Zitationsstil: \citepara{Quelle} => (Quelle Jahreszahl)
\newcommand{\citepara}[1]{(\citealt{#1})}
%Zitationsstil: \pcite{Vgl.}{342}{Quelle} => (Vgl. Quelle, S. 342)
\newcommand*{\pcite}[3]{(\citealt[#1][#2]{#3})}
\newcommand{\citeparapage}[2]{(\citealt[#2]{#1})}
\newcommand{\citeflow}[1]{\citet{#1}}
%\newcommand{\citeflowpage}[2]{\citeauthor[#2]{#1}}

\newcommand*{\fullref}[1]{\hyperref[{#1}]{\autoref*{#1} \nameref*{#1}}}

\newcommand{\paragraphnoindent}[1]{\paragraph{#1}\noindent}

\usepackage{array}
\newcolumntype{L}[1]{>{\raggedright\let\newline\\\arraybackslash\hspace{0pt}}m{#1}}
\newcolumntype{C}[1]{>{\centering\let\newline\\\arraybackslash\hspace{0pt}}m{#1}}
\newcolumntype{R}[1]{>{\raggedleft\let\newline\\\arraybackslash\hspace{0pt}}m{#1}}

%Tabellenkonfig
\definecolor{lightgray}{gray}{0.9}
\let\oldtabular\tabular
\let\endoldtabular\endtabular
\renewenvironment{tabular}{\rowcolors{2}{white}{white}\oldtabular}{\endoldtabular}
%Welche Ebenen bekommen die Linien des Desings in der Überschrift
\setcounter{seclinedepth}{5}

%Gepunktete Linien auch bei Verzeichnissen im Inhaltsverzeichnis
\usepackage{tocstyle}
\newtocstyle[KOMAlike][leaders]{alldotted}{}
\usetocstyle{alldotted}

%Kein Einzug in Verzeichnissen, Abbildung:/Tabelle: als Präfix vor Nummern            
\usepackage[titles]{tocloft}                                         
\setlength{\cftfigindent}{0cm}                                                     
\setlength{\cfttabindent}{0cm} 

\usepackage{tocstyle} 

\makeatletter 
\AfterTOCHead[lof]{% 
	\let\SAVEDNUMBERLINE\tocstyle@numberline 
	\renewcommand*{\tocstyle@numberline}[1]{% 
		\SAVEDNUMBERLINE{\figurename\ #1}% 
	}% 
} 
\AfterTOCHead[lot]{% 
	\let\SAVEDNUMBERLINE\tocstyle@numberline 
	\renewcommand*{\tocstyle@numberline}[1]{% 
		\SAVEDNUMBERLINE{\tablename\ #1}% 
	}% 
} 
\makeatother 

\renewcaptionname{ngerman}\figurename{Abbildung} 
\renewcaptionname{ngerman}\tablename{Tabelle}

\addtocontents{lof}{\protect\def\protect\autodot{: }}
\addtocontents{lot}{\protect\def\protect\autodot{: }}
