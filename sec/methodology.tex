\section{Methodology}
\label{ch:methodology}

This chapter explains the methodology.

\subsection{Dataset}
\label{sec:dataset}

In order to create a dataset suited for the classification task a few aspects had to be considered. \\
As deep learning methods with many parameters that need to be optimized were going to be used for the classification task, many labeled training samples were needed. The dataset should ideally be comprised of equal amounts of synthetically and human generated texts in order to not be biased towards a certain class and thus improve the accuracy of training.

The first idea was to look for already built datasets that are freely available as these would not only reduce 
the time and amount of work needed for the creation of the dataset, but also provide a benchmark against other 
models used on them. Because the examined classification task is fairly novel and powerful language models have 
just started to emerge in recent years, there is a lack of standardized data sets - prior research 
often focused on detection of artificially generated academic papers instead of short texts 
[reference to papers that use academic papers]. Furthermore, the incentive of using metadata related to text 
snippets and inspecting the changes in detection accuracy through it led to the motivation of building a new dataset.

For this purpose, the following sources of human created text were inspected - taking different aspects like 
data availability, extensibility by metadata, potential for use of text generation and minimal overlap with the 
pretrained GPT2 Model into account: Wikipedia, Twitter, reviews (e.g. Amazon, IMDb), Reddit comments, Reuters Corpus.

With everything taken into consideration, Wikipedia articles were chosen as the best fit for this work.
Reasons for this choice were the ease of access, the vast amount of data entries (currently there are more than 6 million articles in the english Wikipedia\footnote{\url{https://en.wikipedia.org/wiki/Wikipedia:Size_comparisons\#Wikipedia}}), the extensibility by metadata such as pageviews or categories and most importantly the fact that the ready-to-use GPT-2 model was explicitly not trained on any Wikipedia article since "it is a common data source for other datasets and could complicate analysis due to overlapping training data with test evalutation tasks."~\footcite{radford2019language}. \\
The reviews and reddit comments datasets were not chosen, because the metadata was not seen as decisive in improving detection quality. The disadvantage of the Reuters Corpus was that the training dataset used for the ready-to-use GPT2 model is comprised of many newspaper sources and thus is more likely to generate results that are rathere similar to their human written counterparts. Twitter was seen as a promising data source, but the access to its public api was shut down which is why it could not be further considered.

\subsubsection{Extraction Process}
\label{sub:extraction_process}

The two usual ways of scraping Wikipedia articles and other data like article metadata or media files from the Wikimedia Foundation\footnote{\url{https://wikimediafoundation.org/}} are through the APIs or the database dumps provided by the MediaWiki platform\footnote{\url{https://www.mediawiki.org/wiki/MediaWiki}}.
Both of these channels were used for different purposes:
The \textbf{database dumps} in the form of 17GB of compressed XML files were firstly downloaded and then titles, clear text and article ids were extracted using a python library called WikiExtractor\footnote{\url{https://github.com/attardi/wikiextractor}}. The output was stored in “.jsonl” (JSON lines) files, where each line denotes a complete JSON object. As the total size of the required metadata is a lot smaller, the \textbf{MediaWiki API} was used to retrieve information such as page views, the latest edit timestamp and categories linked to each article via the extracted article ids. Furthermore the API was used to access the latest news listed in WikiNews to generate exemplary news messages.

The extension of text samples by metadata was done for the purpose of examining three different hypotheses. \textbf{Firstly}, the assumption is being made that the detection rate varies (significantly) across different categories. \textbf{Secondly}, it is expected that the higher the edits and/or views on a page are, the lower the detection rate will be as the human text will be more sophisticated and better worded. \textbf{Thirdly}, the more recent the last edit timestamp of an article is, the lower the detection rate will be as newer information will be less likely to be present in the training data used for \gls{gpt2}. Furthermore, only articles with a minimum length of 1000 characters were taken into consideration as a measure to prevent a decrease in dataset quality (e.g. to filter out entries that only have a redirect notice to another article). For each article entry the last edit timestamp, the aggregated amount of page views over the last 60 days and the namespace was retrieved.

% \begin{itemize}
%     \item{\textbf{Database Dumps:}} Titles, clear text and article ids were extracted using a python library called WikiExtractor\footnote{\url{https://github.com/attardi/wikiextractor}}. MediaWiki provides Wikipedia database dumps in form of compressed XML files can be downloaded and then parsed by WikiExtractor. The output is stored in “.jsonl” (JSON lines) files, where each line denotes a complete JSON object.
%     \item{\textbf{Api:}} The MediaWiki API was used to retrieve metadata such as page views, the latest edit timestamp and categories linked to each article via the extracted page ids, but also to access the latest news listed in WikiNews to generate exemplary news messages.
% \end{itemize}

\subsubsection{Generation Parameter Combination}
\label{sub:generation_parameter_combination}

The output style of GPT2 differs depending on the chosen parameter combination. The parameters that influence the produced output the most are:
\begin{itemize}
    \item Input length
    \item Maximum output length
    \item Temperature - a higher value produces a softer probability distribution over classes which leads to “crazier” or more unlikely text, 
    whereas a lower value does the opposite [cite Hinton Paper no. 23]
    \item Repetition Penalty
\end{itemize}

In order to determine a parameter combination that generates convincing text all possible permutations between the parameters given the values 
listed in [ref table] were used to generate samples for 50 articles. As a metric for evaluation a fine-tuned large (1.5GB weights size as a .pt file) 
RoBERTa-based model with a mixture of temperature 1 and nucleus sampling outputs was chosen. This configuration was elected as it generalizes well 
to outputs generated using different sampling methods [citation gpt2 paper]. On top of that, human evaluation was performed while reading through 
samples created by the best performing parameter combinations according to the RoBERTa evaluation. \\
The main findings were that a higher repetition penalty as well as shorter output length were key factors for better text generation. In 
addition to that, choosing a lower temperature such as [cite Harvard paper] and feeding the LM word split input sentences as opposed to 
character split input sequences (“tennis is a sport” instead of “tennis is a spo”) also improved quality.

[show examples]

The finally selected parameter values were:

\paragraph*{Maximum Input Length - 60 characters}
The input that was fed into GPT-2 XL was the first sequence of plain text in a Wikipedia article, i.e. no infobox or content table text was 
considered. Additionally, before feeding the 60 character String into the LM it was split at the last word. This was done because the quality 
of the generated output tended to increase when input was given in full words rather than in characters and thus having many times split words.

\paragraph*{Maximum Output Length - 50 Tokens}
This corresponds to an average of about 240 - 260 characters per text (in comparison: the max tweet length is 280 characters). As this is a 
size that occurs a lot especially in social media or breaking news headlines (with the subtitle), the focus was placed on shorter text 
snippets [cite techcrunch and socialreport]. 
[insert figure that shows average character length per platform that achieves the best ‘virality’]

\paragraph*{Temperature - 1.0}
Both lowering and increasing (to 0.7 and 1.3) the temperature led to an increased detection of synthetically generated text, which is why 
the temperature was left at its original value.

\paragraph*{Repetition Penalty - 1.3}
As repetitions were not desired in the generated output the repetition penalty was increased from its default value of 1.0 to 1.3. Under the 
assumption that most Wikipedia articles do not contain repetitions (especially in their abstracts) unlike for instance dramatic text, where text 
repetition can be used as a stylistic device.

\paragraph*{Number of beams - 5}
Initially thought of as a parameter that would have a big impact on the quality of the generated text, it was found out that altering the number 
of beams used in the beam sampling strategy when predicting the next most probable tokens was had only little impact on the LM text generation. 
The values chosen ranged between 5 and 10 as this is currently the de facto standard [cite Stanford lectures] in research.


\subsubsection{Data Building}
\label{sub:data_building}

After downloading the Wikipedia dumps the natural text was extracted via the wikiextractor [reference] library (filtering out 
texts with a character length less than 1000) and the metadata was extracted and parsed by using SAX Parser (“Simple Api for XML”) 
[reference]. In order to finalize the data set creation, synthetic text had to be generated for each article’s first 60 characters twice. 
The double generation was performed in order to select the sample that GPT-2 felt more confident on and improve the data set quality.

Given the aforementioned parameter configurations and using the free Google Colaboratory platform [ref Google Colab] which provides 
users with a free GPU (Nvidia Tesla K80 link) the text generation of a single sample took about 5 to 6 seconds. \\
Completing the data set with approximately 100.000 entries, thus, took about 11 days.

The finalized JSON format of a datapoint pair is shown in [Figure 10 - json format of 1 example]. \\

\subsection{Approach}
\label{sec:approach}

\subsubsection{Developed discriminators}
\label{sub:developed_discriminators}

Here I write about developed discriminators.

\subsubsection{Assumptions}
\label{sub:assumptions}

Here I write about assumptions.

This is the approach part.

\subsection{Infrastructure}
\label{sec:infrastructure}

This is the infrastructure part.