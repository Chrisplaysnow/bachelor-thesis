\section{Methodology}
\label{ch:methodology}

This chapter explains the methodology.

\subsection{Dataset}
\label{sec:dataset}

In order to create a dataset suited for the classification task a few aspects had to be considered. \\
As deep learning methods were going to be used for the classification task, many labeled training samples 
were going to be needed [citation for the need of much training data]. The dataset should ideally be comprised 
of equal amounts of synthetically and human generated texts in order to improve the accuracy of training.

The first idea was to look for already built datasets that are freely available as these would not only reduce 
the time and amount of work needed for the creation of the dataset, but also provide a benchmark against other 
models used on them. Because the examined classification task is fairly novel and powerful language models have 
just started to emerge in recent years [citation], there is a lack of standardized data sets - prior research 
often focused on detection of artificially generated academic papers instead of short texts 
[reference to papers that use academic papers]. Furthermore, the incentive of using metadata related to text 
snippets led to the motivation of building a new dataset.

For this purpose, the following sources of human created text were inspected - taking different aspects like 
data availability, extensibility by metadata, potential for use of text generation and minimal overlap with the 
pretrained GPT2-XL Model (1.5B Parameters) into account:

\begin{itemize}
    \item Wikipedia
    \item Twitter
    \item Reviews (e.g. Amazon, IMDb)
    \item Reddit Comments
    \item Reuters Corpus
\end{itemize}

[insert table with different aspects considered]

With everything taken into consideration, Wikipedia articles were chosen as the best fit for this work. 
Especially the fact that OpenAI did explicitly not train their model on any Wikipedia article [cite the gpt2 paper] 
and the ease of accessibility to large amounts of (meta-)data led to this conclusion.

\subsubsection{Data Extraction}
\label{sub:data_extraction}

The two most used ways of scraping Wikipedia articles and other data from the Wikimedia Foundation 
[cite website] are through the APIs [cite] or the database dumps provided by the MediaWiki platform [ref site]. 
Both of these channels were used for different purposes:

\begin{itemize}
    \item{\textbf{Database Dumps:}} Titles, clear text and article id were extracted using a python library called WikiExtractor [ref extractor]. 
    This tool parses the xml-formatted dumps and extracts the aforementioned fields. The output is stored in “.jsonl” (json lines) 
    files [ref json lines], as this is a file format commonly used for nlp data [reference].
    \item{\textbf{Api:}} The MediaWiki Api was used to retrieve metadata such as pageviews, edits, the latest edit timestamp and namespaces (i.e. categories) linked to each article, but also to access the latest news listed in WikiNews to generate exemplary news messages.
\end{itemize}

The extension of text samples by metadata was made in order to examine the following hypotheses:
1. The detection accuracy varies (significantly) across different categories.
2. The higher the edits and/or views on a page are, the lower the detection accuracy will be as the human text will be more sophisticated and better worded.
3. The more recent the last edit timestamp of an article is the lower the detection rate will be as newer information will be less likely to be present in the training data used for GPT-2.

It should be noted that only articles with a minimum length of 1000 characters were taken into consideration in order to filter out many entries that would have diminished the quality of the dataset (e.g. entries that only have a redirect notice [ref]).


\subsection{Approach}
\label{sec:approach}

\subsection{Infrastructure}
\label{sec:infrastructure}